%! Author = ahmedhassanien
%! Date = 5/2/20

% Preamble
\documentclass[aspectratio=169]{beamer}

\usetheme[width=2cm]{Hannover}

% Packages
\usepackage[]{theme/beamercolorthemeowl}
\usepackage{fontawesome}
\usepackage{tikz}
\usetikzlibrary{shapes.geometric}
\usetikzlibrary{arrows.meta,arrows}

\tikzset{
    invisible/.style = {opacity=0},
    visible on/.style = {alt={#1{}{invisible}}},
    alt/.code args = {<#1>#2#3}{\alt<#1>{\pgfkeysalso{#2}}{\pgfkeysalso{#3}}} % \pgfkeysalso doesn't change the path.%------------------------------------------------------------
%This block of code defines the information to appear in the title page
\author[Garage Education] {
    Ahmed Hassanien \& Moustafa Alaa
    \newline
    \faYoutubePlay \space \href{https://youtube.com/c/GarageEducation}{Garage Education}
    \faTwitter \space \href{https://twitter.com/GarageEducation}{@GarageEducation}
}

\date{\today}

\institute{Garage Education}
%\logo{\includegraphics[height=1.5cm]{lion-logo.png}}

%End of title page configuration block
%------------------------------------------------------------
}
\title[Scala Sandbox]{Scala Sandbox}
\subtitle{\textit{Scala SDLC and template}}
% Document
\begin{document}

    \maketitle

%----------------------------------------------------------------------------------------------------------------------%


    \section{Introduction}\label{sec:introduction}
%----------------------------------------------------------------------------------------------------------------------%

    \subsection{Goals}\label{subsec:goals}
    \begin{frame}{Scala Sandbox Goals}
        \begin{itemize}[<+- | alert@+>]
            \item Create Scala SDLC.
            \item Simplify Scala project bootstrapping.
            \item Releasing strategy.
        \end{itemize}
    \end{frame}

    \subsection{Notes}\label{subsec:notes}
    \begin{frame}{Notes}
        \begin{itemize}[<+- | alert@+>]
            \item Maven is an example of a build management tool.
            A similar SDLC is achievable by using other means.
            \item Scala is an example, and we don't expect anyone to be a Scala expert for following this session.
            We are writing simple functions for the demo.
            \item We do bugs to show you the detailed steps to build the project, but we will fix it together
            \item This video focuses on local settings.
            In the next videos, we will show how to configure release this project using the CICD automated process with different tools.
        \end{itemize}
    \end{frame}

    \subsection{Final Takeaways}\label{subsec:final-takeaways}
    \begin{frame}{Notes}
        \begin{itemize}[<+- | alert@+>]
            \item Get your laptop ready with us, pause, and continue following every line.
            \item You need Java (8), and Maven (3.6.x) installed locally.
            \item Check our GitHub repo and articles for more detail.
            \item Ask and share your question on our website.
            \item Please share with us your future expectations about this series.
        \end{itemize}
    \end{frame}


%----------------------------------------------------------------------------------------------------------------------%


    \section{Outline}\label{sec:outline}
%----------------------------------------------------------------------------------------------------------------------%

    \begin{frame}
        \begin{figure}
        \tikzstyle{block} = [rectangle, draw, fill=yellow!90, text width=3cm,minimum height=1em,
        inner sep=0pt,
        text centered, minimum height=1cm,font=\bfseries]

        \begin{tikzpicture}[node distance = 4cm, auto,>=stealth']

        \node [block, visible on=<1->] (b) {Create Scala mvn Proj};
        \node [block, right of=b, visible on=<2->] (c) {Add static analysis tools};
        \node [block, right of=c, visible on=<3->] (d) {Add mvn site reporting};
        \node [block, below of=d, visible on=<4->] (e) {Eliminate boilerplate};
        \node [block, left of=e, visible on=<5->] (f) {Mvn release strategy};
        \node [block, left of=f, visible on=<6->] (g) {GitHub actions};

        \draw [->, visible on=<2->] (b) -- (c);
        \draw [->, visible on=<3->] (c) -- (d);
        \draw [->, visible on=<4->] (d.south) -- (e.north);
        \draw [->, visible on=<5->] (e) -- (f);
        \draw [->, visible on=<6->] (f) -- (g);

        \end{tikzpicture}
        \caption{Outline} \label{fig:M1}
        \end{figure}
    \end{frame}
%----------------------------------------------------------------------------------------------------------------------%
    %\begin{frame}{Outline}
    %    \begin{itemize}[<+- | alert@+>]
    %        \item Create Scala maven project.
    %        \item Add Scala static analysis tools.
    %        \item Add maven site reporting for visualizing your checks.
    %        \item Eliminate maven boilerplate.
    %        \item Maven release strategy.
    %        \item Enable GitHub Actions for Scala maven repository.
    %    \end{itemize}
    %\end{frame}
    %----------------------------------------------------------------------------------------------------------------------%

    %\begin{frame}
    %    \begin{figure}
    %        \tikzstyle{block} = [rectangle, draw, fill=green!60!yellow, text width=3cm,
    %        minimum height=1em,
    %        inner sep=0pt,
    %        text centered, minimum size=1cm,font=\bfseries]
    %
    %        \begin{tikzpicture}[node distance = 4cm, auto,>=stealth']
    %
    %            \node [block] (b) {Create project skeleton};
    %            \node [block, right of=b] (c) {Scala static analysis tools};
    %            \node [block, right of=c] (d) {Compile Scala using mvn};
    %            \node [block, below of=d] (e) {Add sample Scala Test};
    %            \node [block, left of=e] (f) {Configure mvn to run Scala test};
    %            \node [block, left of=f] (g) {Disable compile Java src/tests};
    %
    %            \draw [->] (b) -- (c);
    %            \draw [->] (c) -- (d);
    %            \draw [->] (d.south) -- (e.north);
    %            \draw [->] (e) -- (f);
    %            \draw [->] (f) -- (g);
    %
    %        \end{tikzpicture}
    %        \caption{Scala maven project} \label{fig:M1}
    %    \end{figure}
    %\end{frame}
%----------------------------------------------------------------------------------------------------------------------%
    \subsection{Scala maven project}\label{subsec:scala-maven-project}
    \begin{frame}{Create Scala maven project}
        \begin{itemize}[<+- | alert@+>]
            \item Create project skeleton.
            \item Add sample Scala Class.
            \item Configure maven to compile Scala sources.
            \item Add sample Scala Test.
            \item Configure maven to run Scala test.
            \item Configure maven to avoid compiling java classes and test classes.
        \end{itemize}
    \end{frame}

    \subsection{Static analysis tools}\label{subsec:static-analysis-tools}
    \begin{frame}{Scala static analysis tools}
        \begin{itemize}[<+- | alert@+>]
            \item Add Scala code coverage tool.
            \item Solve the test running twice issue.
            \item Add Scala style tool.
            \item Add FindBugs tool.
        \end{itemize}
    \end{frame}

    \subsection{Maven Site Reporting}\label{subsec:maven-site-reporting}
    \begin{frame}{Maven Site Reporting}
        \begin{itemize}[<+- | alert@+>]
            \item Add Scala Coverage report.
            \item Add FindBugs report.
            \item Add Scala docs report.
        \end{itemize}
    \end{frame}

    \subsection{Eliminate boilerplate}\label{subsec:eliminate-boilerplate}
    \begin{frame}{Eliminate maven boilerplate}
        \begin{itemize}[<+- | alert@+>]
            \item Clean the Scala maven project and use maven BOM and Scala profile.
            \item Add SCM tag to maven (Scala maven project).
        \end{itemize}
    \end{frame}

    \subsection{Maven release strategy}\label{subsec:maven-release-strategy}
    \begin{frame}{Maven release strategy}
        \begin{itemize}[<+- | alert@+>]
            \item Add distribution management tag (Scala maven project).
            \item Create settings.xml to set repository server credentials (Github repository/package).
            \item Add maven release plugin.
            \item Release out first version.
        \end{itemize}
    \end{frame}

    \subsection{GitHub Actions}\label{subsec:github-actions}
    \begin{frame}{GitHub Actions}
        \begin{itemize}[<+- | alert@+>]
            \item Add new GitHub/Workflow/build.yml for Build, Test and Report.
            \item Add deployment with each develop push.
            \item Add release with each master push (Merge request from develop to master).
        \end{itemize}
    \end{frame}

\end{document}